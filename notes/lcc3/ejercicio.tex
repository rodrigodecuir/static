\documentclass{article}
\usepackage[spanish,es-table]{babel}
\usepackage{enumitem}
\usepackage{float}
\usepackage{amsmath}
\usepackage{graphicx}

\title{Ejercicios}
\author{Teresa B. \\ Siguiente nombre}
\date{\today}

\begin{document}
\maketitle

Mi primer documento en \LaTeX!

Este será nuestro segundo párrafo.\\
Esta será una nueva línea.

\section{Primera sección}

% Mi primer comentario

\textbf{Texto en negrita}\\
\textit{Texto en cursiva}\\
\texttt{Texto en tipo máquina}\\
\textsc{Texto en mayúsculas eScAlOnAdAs}

{\bf Un texto con más palabras}

Tamaños de letras

{\tiny Texto muy pequeño}

{\small Texto pequeño}

{\Large Texto grande}

{\Huge Texto demasiado grande}




\subsection*{Subsección sin numeración}

\section{Segunda subsección}
\subsubsection{Algo}

% Ejercicio

% Crea un archivo ejercicio1.tex de clase article y
% realiza lo siguiente:

% 1. Define el tipo de clase, título, autor, fecha
% Título = Ejercicio 1, Autor = Nombre, Fecha = Agosto 2024
% Agrega una sección llamada Introducción.
% En ella primero van agregar un comentario (Lo que ustedes quieran agregar) despuésel van a escribir tres párrafos:
% El primero con letra en itálicas y tamaño small.
{\it\small texto pequeño}
% El segundo con letras en negritas y tamaño Huge.
{\bf\Huge Patito feo}

\section{Ejercicios}
\subsection{Listas}

\begin{itemize}
\item Elemento 1
\item Elemento 2
  \item Elemento 3
  \end{itemize}

\begin{itemize}
\item Elemento 1
  \begin{itemize}
  \item Subelemento 1
  \item Subelemento 2
      \item Subelemento 3
  \end{itemize}
\item Elemento 2
  \item Elemento 3
  \end{itemize}
  
  \begin{enumerate}
  \item Elem 1
  \item Elem 2
    \item Elem 3
    \end{enumerate}

      \begin{enumerate}[label=\alph*)]
  \item Elem 1
  \item Elem 2
    \begin{itemize}
    \item Elemento 2.1
    \item Elemento 2.2
    \end{itemize}
    \item Elem 3
    \end{enumerate}

    \begin{enumerate}
    \item E1
      \begin{enumerate}
      \item E1.1
      \end{enumerate}
     \end{enumerate}

     \newlist{contract}{enumerate}{5}
     \setlist[contract]{label*=\arabic*.}
     \setlistdepth{5}

     \begin{contract}
     \item nivel 1
       \begin{contract}
       \item nivel 2
         \begin{contract}
         \item nivel 3
         \end{contract}
       \end{contract}
     \end{contract}

     \subsection{Modo matemático}

     \subsection{Modo en Línea}

     Primera: $n_i = 2x$ \\
     Segunda: \(n_i = 2x\)\\
     Tercera: \begin{math} n_{3i} = 2x\end{math}
     \subsection{Modo despliegue}

     \[n_i = 2x\]
     \begin{equation}
       n_i = 2x
       \end{equation}

       $$n_i = 2x$$
\begin{equation*}
       n_i = 2x
       \end{equation*}

       \subsection{Figuras}

       % h - aquí, b - abajo, t - arriba, p - otra página
       % centering - centrar, \raggedleft - Alineado a la izquierda
       % raggedright - Alineado a la derecha

       % Opciones
       % width - Ancho, height - Altura, scale - escala, angle - ángulo
       \begin{figure}[hb]
         \centering
         \includegraphics[width = 0.5\linewidth]{gato.jpg}
         \caption{Gato}
         \label{fig:gato}
       \end{figure}

       \includegraphics[scale = 0.5, angle=90]{gato.jpg}

       \subsection{Tablas}

       \begin{tabular}{ | c | c |}
         \hline
         Lenguaje & Programas\\
         \hline
         Java & 8 \\
         Python & 10 \\
         Haskell & 3 \\
         \hline
       \end{tabular}

       \begin{table}[H]
         \centering
          \begin{tabular}{ | c | c |}
         \hline
         Lenguaje & Programas\\
         \hline
         Java & 8 \\
         Python & 10 \\
         Haskell & 3 \\
         \hline
          \end{tabular}
          \caption{Numero de programas}
          \label{tab:programas}
    \end{table}
\end{document}
